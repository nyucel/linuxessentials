\chapter{Dosya Sistemi}
\paragraph{Amaçlar}{
\begin{itemize}
 \item “dosya” ve “sistem dosya” kavramların anlamak
 \item Farklı dosya türlerin tanıma
 \item Linux sistemin ağaç dizi yolların öğrenmek
 \item Dizi ağaç içerisinde harici dosya nasıl entegre edilmesini öğrenmek
 \end{itemize}}
\paragraph{Önceden Bilinmesi Gerekenler}
\begin{itemize}
 \item Linux temel bilgileri (Önceki Konulardan)
 \item Dosyalar ve Dizinleri işlemesi (Bölüm 6)
 \end{itemize}

\paragraph{}{
\begin {table}[H]
\caption {Linux Dosya Türleri} \label{tab:title} 
\begin{tabular}{l c l l}
\hline
Tür & ls -l& ls -F& Nasıl oluşturulur\\
\hline
düz metin & -&  isim& çeşitli programlar\\
dizin & d & isim/ & mkdir \\
sembolik link & l & isim@ & ln -s \\
aygıt dosyası & b veya c & isim & mknod \\
FIFO (pipe) & p & name \textbar & mkfifo \\
UNIX domain socket & s & isim= & komut yok\\
\hline
\end{tabular}
\end {table}
}

\begin{section}{Terimler}
Genelde dosya, verilerin kendi içerisinde bulunan toplamlarıdır. Dosya içindeki veri türlerine bağlı herhangi bir kısıtlama yoktur; bir dosya metin birkaç harftan oluşabilir veya kullanıcın tam iş hayatın birden çok megabyte içeren arşivden oluşur. Dosyalar düz metin içermeleri gerekmez. Görüntü, Ses,.. çalişabilir uygulamalar ve diğer pek çok dosyalar Bir depo üzerine yerleştirilir. Bir dosya veri türünü tahmin etmek için dosyanın içinde bulunan dosya komutun kulanabilir:

\begin{verbatim}
$ file /bin/ls /usr/bin/groups /etc/passwd
/bin/ls: ELF 32-bit LSB executable, Intel 80386,
  version 1 (SYSV), for GNU/Linux 2.4.1,
dynamically linked (uses shared libs), for GNU/Linux 2.4.1, stripped
/usr/bin/groups: Bourne shell script text executable
/etc/passwd: ASCII text
\end{verbatim}

dosya /usr/share/file alt dizinindeki kurallara uygun dosya sistemini tahmin eder.
yönetici /usr/share/file/magic alt dizini kuralların bulunduğu bir metin dosyası
bulundurur.. Kendi kurallarınızı /etc/magic alt dizinine koymak şartıyla
tanımlayabilirsiniz. Detaylar için magic(5)'e bakınız. Uygun bir şekilde işleyebilmesi için bir Linux sistemi binlerce farklı dosyaya ihtiyaç duyar. Bunlar sistemin çeşitli kullanıcıları tarafından oluşturulmuş ve sahip olunan çeşitli
dosyalardır.

\end{section}
\begin{section}{Dosya Türleri}


\end{section}
\begin{section}{Linux Dizin Ağacı}


\end{section}
\begin{section}{Dizi Ağacı ve Dosya Sistemleri}


\end{section}