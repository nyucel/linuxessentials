\chapter{Dosya Sistemi}
\paragraph{Amaçlar}{
\begin{itemize}
 \item “dosya” ve “sistem dosya” kavramların anlamak
 \item Farklı dosya türlerin tanıma
 \item Linux sistemin ağaç dizi yolların öğrenmek
 \item Dizi ağaç içerisinde harici dosya nasıl entegre edilmesini öğrenmek
 \end{itemize}}
\paragraph{Önceden Bilinmesi Gerekenler}
\begin{itemize}
 \item Linux temel bilgileri (Önceki Konulardan)
 \item Dosyalar ve Dizinleri işlemesi (Bölüm 6)
 \end{itemize}

\paragraph{}{
\begin {table}[H]
\caption {Linux Dosya Türleri} \label{tab:title} 
\begin{tabular}{l c l l}
\hline
Tür & ls -l& ls -F& Nasıl oluşturulur\\
\hline
düz metin & -&  isim& çeşitli programlar\\
dizin & d & isim/ & mkdir \\
sembolik link & l & isim@ & ln -s \\
aygıt dosyası & b veya c & isim & mknod \\
FIFO (pipe) & p & name \textbar & mkfifo \\
UNIX domain socket & s & isim= & komut yok\\
\hline
\end{tabular}
\end {table}
}

\begin{section}{Terimler}

\begin{verbatim}
$ file /bin/ls /usr/bin/groups /etc/passwd
/bin/ls: ELF 32-bit LSB executable, Intel 80386,
  version 1 (SYSV), for GNU/Linux 2.4.1,
dynamically linked (uses shared libs), for GNU/Linux 2.4.1, stripped
/usr/bin/groups: Bourne shell script text executable
/etc/passwd: ASCII text
\end{verbatim}

\end{section}
\begin{section}{Dosya Türleri}


\end{section}
\begin{section}{Linux Dizin Ağacı}


\end{section}
\begin{section}{Dizi Ağacı ve Dosya Sistemleri}


\end{section}